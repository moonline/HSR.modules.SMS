%Pakete;
%A4, Report, 12pt
\documentclass[ngerman,a4paper,12pt]{scrreprt}
\usepackage[a4paper, right=20mm, left=20mm,top=20mm, bottom=30mm, marginparsep=5mm, marginparwidth=5mm, headheight=7mm, headsep=15mm,footskip=15mm]{geometry}

%Papierausrichtungen
\usepackage{pdflscape}
\usepackage{lscape}

%Deutsche Umlaute, Schriftart, Deutsche Bezeichnungen
\usepackage[utf8]{inputenc}
\usepackage[T1]{fontenc}
\usepackage[ngerman]{babel}

%quellcode
\usepackage{listings}

%tabellen
\usepackage{tabularx}

%listen und aufzählungen
\usepackage{paralist}

%farben
\usepackage[svgnames,table,hyperref]{xcolor}

%symbole
\usepackage{latexsym,textcomp}

%font
\usepackage{helvet}
\renewcommand{\familydefault}{\sfdefault}

%Abkürzungsverzeichnisse
\usepackage[printonlyused]{acronym}

%Bilder
\usepackage{graphicx} %Bilder
\usepackage{float}	  %"Floating" Objects, Bilder, Tabellen...
\usepackage[space]{grffile} %Leerzechen Problem bei includegraphics
\usepackage{wallpaper} %Seitenhintergrund setzen
\usepackage{transparent} %Transparenz

%for
\usepackage{forloop}
\usepackage{ifthen}

%Dokumenteigenschaften
\title{Repetitionsfragen SMS}
\author{Tobias Blaser}
\date{\today{}, Rapperswil}


%Kopf- /Fusszeile
\usepackage{fancyhdr}
\usepackage{lastpage}

\pagestyle{fancy}
	\fancyhf{} %alle Kopf- und Fußzeilenfelder bereinigen
	\renewcommand{\headrulewidth}{0pt} %obere Trennlinie
	\fancyfoot[L]{Seite \thepage/\pageref{LastPage}} %Fusszeile mitte
	\fancyfoot[R]{\today{}} %Fusszeile rechts
	\renewcommand{\footrulewidth}{0.4pt} %untere Trennlinie

%Kopf-/ Fusszeile auf chapter page
\fancypagestyle{plain} {
	\fancyhf{} %alle Kopf- und Fußzeilenfelder bereinigen
	\renewcommand{\headrulewidth}{0pt} %obere Trennlinie
	\fancyfoot[L]{Seite \thepage/\pageref{LastPage}} %Fusszeile mitte
	\fancyfoot[R]{\today{}} %Fusszeile rechts
	\renewcommand{\footrulewidth}{0.4pt} %untere Trennlinie
}

\usepackage{changepage}

% Abkürzungen für Kapitel, Titel und Listen
\input{commands/shortcutsListAndChapter}
\input{commands/TextStructuringBoxes}

%links, verlinktes Inhaltsverzeichnis, PDF Inhaltsverzeichnis
\usepackage[bookmarks=true,
bookmarksopen=true,
bookmarksnumbered=true,
breaklinks=true,
colorlinks=true,
linkcolor=black,
anchorcolor=black,
citecolor=black,
filecolor=black,
menucolor=black,
pagecolor=black,
urlcolor=black
]{hyperref} % Paket muss unbedingt als letzes eingebunden werden!

\usepackage{graphicx}
\begin{document}

% Inhaltsverzeichnis
\tableofcontents
\clearpage

\ch{Sockets}
\ol
	\li Nennen Sie Gründe, die für ein Simulationsprojekt sprechen.
	\li Warum / Wozu wird simulaion verwendet?
	\li Was ist Simulation?
	\li Was ist ein Model was ein System und was sind Experimente?
	\li Was sind physikalische und logische Modelle? Wo ist der Unterschied?
	\li Erklären Sie die folgenden Simulationsklassen: dynamic, discrete-change, stochastic.
	\li Was ist event driven simulation?
	\li Erklären Sie den Unterschied zwischen diskreter und kontinuierlicher Simulation.
	\li Was ist statische und was dynamische Simulation?
	\li Was sind equidistante Systemsprünge?
	\li Nennen Sie einige Fragen, die eine Simulation eines Systems beantworten könnte.
	\li Nennen Sie einige Key Performance Indicators.
	\li Nennen Sie verschieden ausgefeilte/aufbereitete Simulationsarten, die bei einem Projekt gefordert sein können.
	\li Abb. \ref{simProc}: Erklären Sie den Simulationsprozess.
	\img{img/v2.1.jpg}{}{0.75}{simProc}
	\li Abb. \ref{simProc2}: Erklären Sie die Grafik.
	\img{img/v2.2.jpg}{}{0.75}{simProc2}
\olS


\ch{Event Driven Architecture}
\olR
	\li Wodurch zeichnen sich Ereignisgesteuerte Prozesse aus?
	\li Was ist ein Ereignis?
	\li Erklären Sie den Ereigniszyklus (.. \ra Erkennen \ra Verarbeiten \ra Reagieren \ra ..)
	\li Nennen Sie einige Eigenschaften der folgenden Begriffe: Ereignisquelle, Ereignisobjekt/Nachricht, Ereignissenke
	\li Stellen Sie synchrone Interaktion asynchroner gegenüber und nennen sie Konsequenzen für die komplette Interaktion in einem System.
	\li 
\olS





\end{document}
