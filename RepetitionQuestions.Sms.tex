%Pakete;
%A4, Report, 12pt
\documentclass[ngerman,a4paper,12pt]{scrreprt}
\usepackage[a4paper, right=20mm, left=20mm,top=20mm, bottom=30mm, marginparsep=5mm, marginparwidth=5mm, headheight=7mm, headsep=15mm,footskip=15mm]{geometry}

%Papierausrichtungen
\usepackage{pdflscape}
\usepackage{lscape}

%Deutsche Umlaute, Schriftart, Deutsche Bezeichnungen
\usepackage[utf8]{inputenc}
\usepackage[T1]{fontenc}
\usepackage[ngerman]{babel}

%quellcode
\usepackage{listings}

%tabellen
\usepackage{tabularx}

%listen und aufzählungen
\usepackage{paralist}

%farben
\usepackage[svgnames,table,hyperref]{xcolor}

%symbole
\usepackage{latexsym,textcomp}

%font
\usepackage{helvet}
\renewcommand{\familydefault}{\sfdefault}

%Abkürzungsverzeichnisse
\usepackage[printonlyused]{acronym}

%Bilder
\usepackage{graphicx} %Bilder
\usepackage{float}	  %"Floating" Objects, Bilder, Tabellen...
\usepackage[space]{grffile} %Leerzechen Problem bei includegraphics
\usepackage{wallpaper} %Seitenhintergrund setzen
\usepackage{transparent} %Transparenz

%for
\usepackage{forloop}
\usepackage{ifthen}

%Dokumenteigenschaften
\title{Repetitionsfragen SMS}
\author{Tobias Blaser}
\date{\today{}, Rapperswil}


%Kopf- /Fusszeile
\usepackage{fancyhdr}
\usepackage{lastpage}

\pagestyle{fancy}
	\fancyhf{} %alle Kopf- und Fußzeilenfelder bereinigen
	\renewcommand{\headrulewidth}{0pt} %obere Trennlinie
	\fancyfoot[L]{Seite \thepage/\pageref{LastPage}} %Fusszeile mitte
	\fancyfoot[R]{\today{}} %Fusszeile rechts
	\renewcommand{\footrulewidth}{0.4pt} %untere Trennlinie

%Kopf-/ Fusszeile auf chapter page
\fancypagestyle{plain} {
	\fancyhf{} %alle Kopf- und Fußzeilenfelder bereinigen
	\renewcommand{\headrulewidth}{0pt} %obere Trennlinie
	\fancyfoot[L]{Seite \thepage/\pageref{LastPage}} %Fusszeile mitte
	\fancyfoot[R]{\today{}} %Fusszeile rechts
	\renewcommand{\footrulewidth}{0.4pt} %untere Trennlinie
}

\usepackage{changepage}

% Abkürzungen für Kapitel, Titel und Listen
\input{commands/shortcutsListAndChapter}
\input{commands/TextStructuringBoxes}

%links, verlinktes Inhaltsverzeichnis, PDF Inhaltsverzeichnis
\usepackage[bookmarks=true,
bookmarksopen=true,
bookmarksnumbered=true,
breaklinks=true,
colorlinks=true,
linkcolor=black,
anchorcolor=black,
citecolor=black,
filecolor=black,
menucolor=black,
pagecolor=black,
urlcolor=black
]{hyperref} % Paket muss unbedingt als letzes eingebunden werden!

\usepackage{graphicx}
\begin{document}

% Inhaltsverzeichnis
\tableofcontents
\clearpage

\ch{Sockets}
\ol
	\li Nennen Sie Gründe, die für ein Simulationsprojekt sprechen.
	\li Warum / Wozu wird simulaion verwendet?
	\li Was ist Simulation?
	\li Was ist ein Model was ein System und was sind Experimente?
	\li Was sind physikalische und logische Modelle? Wo ist der Unterschied?
	\li Erklären Sie die folgenden Simulationsklassen: dynamic, discrete-change, stochastic.
	\li Was ist event driven simulation?
	\li Erklären Sie den Unterschied zwischen diskreter und kontinuierlicher Simulation.
	\li Was ist statische und was dynamische Simulation?
	\li Was sind equidistante Systemsprünge?
	\li Nennen Sie einige Fragen, die eine Simulation eines Systems beantworten könnte.
	\li Nennen Sie einige Key Performance Indicators.
	\li Nennen Sie verschieden ausgefeilte/aufbereitete Simulationsarten, die bei einem Projekt gefordert sein können.
	\li Abb. \ref{simProc}: Erklären Sie den Simulationsprozess.
	\img{img/v2.1.jpg}{}{0.75}{simProc}
	\li Abb. \ref{simProc2}: Erklären Sie die Grafik.
	\img{img/v2.2.jpg}{}{0.75}{simProc2}
\olS


\ch{Event Driven Architecture}
\olR
	\li Wodurch zeichnen sich Ereignisgesteuerte Prozesse aus?
	\li Was ist ein Ereignis?
	\li Erklären Sie den Ereigniszyklus (.. \ra Erkennen \ra Verarbeiten \ra Reagieren \ra ..)
	\li Nennen Sie einige Eigenschaften der folgenden Begriffe: Ereignisquelle, Ereignisobjekt/Nachricht, Ereignissenke
	\li Stellen Sie synchrone Interaktion asynchroner gegenüber und nennen sie Konsequenzen für die komplette Interaktion in einem System.
	\li Skizzieren Sie das Observer Pattern mit einer Middleware
	\li Was ist 'Complex Event Processing' und 'Event Stream Processing'? Wi ist der UNterschied?
	\li Abb. \ref{edacep}: Erklären Sie die Grafik.
		\img{img/v3.3.jpg}{}{0.75}{}
	\li Welche Probleme/Defizite besitzt der Ereignisgesteuerte Ansatz?
	\li Was ist die Modelluhr und was die Reale Uhr? Wozu benötigen Sie eine Modelluhr?
	\li Erklären Sie, wie parallele Ereignisse in Realzeit und in Modellzeit abgearbeitet werden.
	\li Was ist das Dualitätsprinzip bez. Modelluhr?
	\li Wann bietet sich ein Eventgetriebenes Modell an?
	\li Was ist der Unterschied zwischen Rechenzeit / Realer Zeit / Modelzeit / Simulationszeit?
	\li Parallele Prozesse können sich gegenseitig Beeinflussen. Erklären Sie die beiden Methoden wie mit diesen Beeinflussungen umgegangen werden kann.
\olS


\ch{Warteschlangen}
\olR
	\li Erklären Sie das Notationssystem von Warteschlangen am Beispiel 'M/M/1/$\infty$/$\infty$/fifo'.
	\li Wie funktioniert die Simple Queue? Wie berechnen Sie die mittlere Aufenthaltszeit im System?
	\li Erklären Sie das Little's Gesetz.
	\li Was passiert bei einer Simple Queue, die zu 100\% ausgelastet ist? Skizzieren Sie die Residenztime in Abhängigkeit der Auslastung.
	\li Wie funktioniert das Twin Center?  Wie berechnen Sie die mittlere Aufenthaltszeit im System?
	\li Wie funktioniert der Dual Server?  Wie berechnen Sie die mittlere Aufenthaltszeit im System? Wie berechnen Sie die mittlere Aufenthaltszeit in einem Multi Server Dingle Que (M/M/n) System?
	\li Zeichnen Sie die Residenzzeit/Auslastungsfunktion für Twin Center und Dual Server. Wie und warum unterscheiden sich die Kurven?
	\li Wie funktioniert das Feedback Center? Wie berechnen Sie die mittlere Aufenthaltszeit im System?
	\li Was ist ein Closed Queueing Center? Wie berechnen Sie die mittlere Aufenthaltszeit im System?
\olS


\ch{Input Datenanalyse}
\olp{
	\li Was ist eine Zufallsvariable und was eine Verteilungsdichtefunktion?
	\li Erklären Sie den Unterschied zwischen einer diskreten und einer	kontinuierlichen Verteilungsfunktion
	\li Was ist eine Wahrscheinlichkeitsfunktion und was eine Verteilungs- und was eine Summenfunktion?
	\li Was ist der Unterschied zwischen einer Verteilungsfunktion und einer Dichtefunktion?
	\li Erklären Sie die Begriffe Erwartungswert, Standardabweichung und Varianz.
	\li Nennen Sie einige Eigenschaften eines dynamischen Simulationsexperiments.
	\li Skizzieren Sie die folgenden Funktionen und geben Sie an, ob es sich um diskrete oder kontinuierliche Funktionen handel:
		\oli{
			\li Bernoulli
			\li Binominal
			\li Beta
			\li Exponential
			\li Gamma
			\li Geometric
			\li Lognormal
			\li Normal
			\li Poison
			\li Step
			\li Triangular
			\li Uniform
			\li Weilbull
		}
	\li Welche Testverfahren gibt es für Verteilungsfunktionen?
	\li Was ist eine Zufallsvariable?
	\li Geben sie Beispiele für stetige und diskrete Zufallsvariablen an!
	\li Welcher Unterschied besteht zwischen Verteilungsdichte und Verteilungsfunktion
	\li Was versteht man bei pseudo Zufallszahlen unter der seed?
	\li Was versteht man bei Zufallszahlen unter Sequenzlänge und was bedeutet das für die Simulation?
	\li Wie können sie mit einer gleichverteilten Zufallsvariablen Zufallszahlen anderer Verteilungen erzeugen?
	\li Warum werden in der Simulation Pseudo-Zufallszahlen unterschiedlicher Verteilungen benötigt?
	\li Hat die Auswahl der Verteilungsfunktion einen grossen Einfluss auf das Resultat der
	\li Simulation? Geben sie ein einleuchtendes Beispiel an.
}

\se{Zufallsvariablen}
\olp{
	\li Welche Eigenschaften müssen Zufallszahlen besitzen, damit sie für Simulationen verwendet werden können?
	\li Warum können für Simulationsexperimente keine ´´echten'' Zufallszahlen verwendet werden?
	\li Was ist ein RNG?
	\li Wie werden gleichverteilte Zufallszahlen erzeugt?
	\li Wie werden negativ exp. verteilte Zufallszahlen erzeugt?
}






\end{document}
